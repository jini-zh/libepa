\documentclass[a4paper,12pt]{article}
\usepackage[left=20mm,right=20mm,top=20mm,bottom=30mm]{geometry}
\usepackage{microtype}
\usepackage{amsmath}
\usepackage{mathtools}
\usepackage{hyperref}
\usepackage{tikz}

\hypersetup{
  colorlinks,
  linkcolor={blue!50!black},
  citecolor={blue!50!black},
  urlcolor={blue!50!black}
}

\begin{document}

\section{Form factors}

Form factor of a point particle:
\begin{equation}
  F_0(Q^2) = 1,
\end{equation}
where $Q^2$ is the photon 3-momentum squared.

Monopole form factor:
\begin{equation}
  F_1(Q^2) = \frac{1}{1 + \frac{Q^2}{\Lambda^2}}.
\end{equation}

Dipole form factor:
\begin{equation}
  F_2(Q^2) = \frac{1}{\left( 1 + \frac{Q^2}{\Lambda^2} \right)^2}.
\end{equation}

\section{EPA spectra}

Equivalent photon spectrum integrated across transversal plane:
\begin{equation}
  n(\omega)
  = \frac{2 Z^2 \alpha}{\pi \omega}
    \int\limits_0^\infty
      \left[
        \frac{F(q_\perp^2 + (\omega / \gamma)^2)}
             {q_\perp^2 + (\omega / \gamma)^2}
      \right]^2
      q_\perp^3
      \, \mathrm{d} q_\perp,
\end{equation}
where $\omega$ is the photon energy $Z e$ is the charge of the particle,
$\gamma$ is its Lorentz factor, $\alpha$ is the fine structure constant,
$q_\perp$ is the photon transverse momentum, $F(Q^2)$ is the particle
electromagnetic form factor.

Equivalent photon spectrum of a point particle is logarithmically divergent.

Equivalent photon spectrum for monopole form factor:
\begin{equation}
  n_1(\omega)
  = \frac{Z^2 \alpha}{\pi \omega}
    \left[ (2 a + 1) \ln \left(1 + \frac{1}{a} \right) - 2 \right],
\end{equation}
where $a = (\omega / \Lambda \gamma)^2$.

Equivalent photon spectrum for dipole form factor:
\begin{equation}
  n_2(\omega)
  = \frac{Z^2 \alpha}{\pi \omega}
    \left[
      (4 a + 1) \ln \left(1 + \frac{1}{a} \right)
      - \frac{24 a^2 + 42 a + 17}{6 (a + 1)^2}
    \right],
\end{equation}
where $a = (\omega / \Lambda \gamma)^2$.

Equivalent photon spectrum at distance $b$ from the source particle:
\begin{equation}
  n(b, \omega)
  = \frac{Z^2 \alpha}{\pi^2 \omega}
    \left[
      \int\limits_0^\infty
        \frac{F(q_\perp^2 + (\omega / \gamma)^2)}
             {q_\perp^2 + (\omega / \gamma)^2}
        J_1(b q_\perp)
        \, q_\perp^2
        \, \mathrm{d} q_\perp
    \right]^2,
\end{equation}
where $J_1(x)$ is the Bessel function of the first kind.

Equivalent photon spectrum of point-like particle:
\begin{equation}
  n_0(b, \omega)
  = \frac{Z^2 \alpha \omega}{\pi^2 \gamma^2}
    K_1^2 \left( \frac{b \omega}{\gamma} \right),
\end{equation}
where $K_1(x)$ is the modified Bessel function of the second kind (the Macdonald
function).

Equivalent photon spectrum for monopole form factor:
\begin{equation}
  n_1(b, \omega)
  = \frac{Z^2 \alpha}{\pi^2 \omega}
    \left[
        \frac{\omega}{\gamma} K_1 \left( \frac{b \omega}{\gamma} \right)
      - \sqrt{\Lambda^2 + \left( \frac{\omega}{\gamma} \right)^2}
        K_1 \left(
          b \sqrt{\Lambda^2 + \left( \frac{\omega}{\gamma} \right)^2}
        \right)
    \right]^2.
\end{equation}

Equivalent photon spectrum for dipole form factor:
\begin{multline}
  n_2(b, \omega)
  = \frac{Z^2 \alpha}{\pi^2 \omega}
    \left[
        \frac{\omega}{\gamma} K_1 \left( \frac{b \omega}{\gamma} \right)
      - \sqrt{\Lambda^2 + \left( \frac{\omega}{\gamma} \right)^2}
        K_1 \left(
          b \sqrt{\Lambda^2 + \left( \frac{\omega}{\gamma} \right)^2}
        \right)
  \right. \\ \left.
      - \frac{b \Lambda^2}{2}
        K_0 \left(
          b \sqrt{\Lambda^2 + \left( \frac{\omega}{\gamma} \right)^2}
        \right)
    \right].
\end{multline}

Equivalent photon spectrum at distance $b$ with the form factor given as a set
of points $(Q_i^2, F(Q_i)^2)$ is calculated with the help of the following
relations:
\begin{gather}
  n(b, \omega)
  = \frac{Z^2 \alpha}{\pi^2 \omega}
    \left[
      \sum\limits_i
        \int\limits_{\sqrt{Q_i^2 - \left( \frac{\omega}{\gamma} \right)^2}}
                   ^{\sqrt{Q_{i+1}^2 - \left( \frac{\omega}{\gamma} \right)^2}}
          \left(
              A_i q_\perp^2
            + B_i
            - \frac{B_i \ (\omega / \gamma)^2}{q_\perp^2 + (\omega / \gamma)^2}
          \right)
          J_1(b q_\perp)
    \right]^2,
  \\
  A_i = \frac{F(Q_{i+1}^2) - F(Q_i^2)}{Q_{i+1}^2 - Q_i^2},
  \ 
  B_i = F(Q_i^2) - A Q_i^2,
  \\
  \int x^2 J_1(a x) \mathrm{d} x = \frac{x^2}{a} J_2(a x) + \text{const}, \\
  \int J_1(a x) \mathrm{d} x = -\frac{1}{a} J_0(x) + \text{const}.
\end{gather}

\section{Luminosity}

Photon-photon luminosity in ultraperipheral collision of particles $A$ and $B$
differentiated with respect to rapidity of the system $y = \tfrac12 \ln
\tfrac{\omega_1}{\omega_2}$ with non-electromagnetic interactions neglected:
\begin{equation}
  \frac{\mathrm{d} L_{AB}}{\mathrm{d} s}
  = \frac14
    \int\limits_{-\infty}^{\infty}
      n_A \left( \frac{\sqrt{s}}{2} \mathrm{e}^y \right)
      n_B \left( \frac{\sqrt{s}}{2} \mathrm{e}^{-y} \right)
    \mathrm{d} y,
\end{equation}
where $n_A$ and $n_B$ are the equivalent photon spectra of the particles $A$ and
$B$, $s = 4 \omega_1 \omega_2$ is the invariant mass of the photons. Calculation
convergence is better when we change the integration variable to $x =
\frac{\omega_1}{\omega_2} = \exp{2 y}$:
\begin{equation}
  \frac{\mathrm{d} L_{AB}}{\mathrm{d} s}
  = \frac18
    \int\limits_0^\infty
       n_A \left( \sqrt{\frac{sx}{4}} \right)
       n_B \left( \sqrt{\frac{s}{4x}} \right)
       \frac{\mathrm{d} x}{x}.
\end{equation}

Simplification when $A = B$:
\begin{equation}
  \frac{\mathrm {d} L}{\mathrm{d} s}
  = 2 \int\limits_0^\infty
        n \left( \frac{\sqrt{s}}{2} \mathrm{e}^y \right)
        n \left( \frac{\sqrt{s}}{2} \mathrm{e}^{-y} \right)
      \mathrm{d} y
  = \frac14
    \int\limits_0^1
      n \left( \sqrt{\frac{sx}{4}} \right)
      n \left( \sqrt{\frac{s}{4x}} \right)
      \frac{\mathrm{d} x}{x}.
\end{equation}

Photon-photon luminosity with non-electromagnetic interactions respected:
\begin{equation}
  \begin{aligned}
    \frac{\mathrm{d} L_\parallel}{\mathrm{d} s}
    &= \frac14
       \int\limits_{-\infty}^\infty \mathrm{d} y
       \int\limits \mathrm{d}^2 b_1
       \int\limits \mathrm{d}^2 b_2
       \, n \left( b_1, \tfrac{\sqrt{s}}{2} \, \mathrm{e}^y \right)
       \, n \left( b_2, \tfrac{\sqrt{s}}{2} \, \mathrm{e}^{-y} \right)
       \, P(b)
       \, \cos^2 \varphi,
    \\
    \frac{\mathrm{d} L_\perp}{\mathrm{d} s}
    &= \frac14
       \int\limits_{-\infty}^\infty \mathrm{d} y
       \int\limits \mathrm{d}^2 b_1
       \int\limits \mathrm{d}^2 b_2
       \, n \left( b_1, \tfrac{\sqrt{s}}{2} \, \mathrm{e}^y \right)
       \, n \left( b_2, \tfrac{\sqrt{s}}{2} \, \mathrm{e}^{-y} \right)
       \, P(b)
       \, \sin^2 \varphi,
  \end{aligned}
\end{equation}
where $\mathrm{d} L_\parallel / \mathrm{d} s$ is the luminosity of photons with
parallel polarizations, $\mathrm{d} L_\perp / \mathrm{d} s$ is the luminosity
of photons with perpendicular polarizations, $P(b)$ is the probability for the
particles to survive in an ultraperipheral collision with the impact parameter
$b$ (the probability to avoid non-electromagnetic interactions), $b =
\sqrt{b_1^2 + b_2^2 - 2 b_1 b_2 \cos \varphi}$. The integration can be
rearranged as follows:
\begin{equation}
  \begin{aligned}
    \frac{\mathrm{d} L_\perp }{\mathrm{d} s}
    &= \begin{multlined}[t]
         \frac{\pi}{4}
         \int\limits_0^\infty
           b_1
           \int\limits_0^\infty
             b_2
             \int\limits_0^\infty
               n_A \left( b_1, \sqrt{\frac{sx}{4}} \right)
               n_B \left( b_2, \sqrt{\frac{s}{4x}} \right)
             \frac{\mathrm{d} x}{x}
      \times {} \\
             \int\limits_0^{2 \pi}
               P \left( \sqrt{b_1^2 + b_2^2 - 2 b_1 b_2 \cos \varphi} \right)
               \cos^2 \varphi
             \, \mathrm{d} \varphi
           \, \mathrm{d} b_2
         \, \mathrm{d} b_1,
       \end{multlined}
    \\
    \frac{\mathrm{d} L_\parallel }{\mathrm{d} s}
    &= \begin{multlined}[t]
         \frac{\pi}{4}
         \int\limits_0^\infty
           b_1
           \int\limits_0^\infty
             b_2
             \int\limits_0^\infty
               n_A \left( b_1, \sqrt{\frac{sx}{4}} \right)
               n_B \left( b_2, \sqrt{\frac{s}{4x}} \right)
             \frac{\mathrm{d} x}{x}
      \times {} \\
             \int\limits_0^{2 \pi}
               P \left( \sqrt{b_1^2 + b_2^2 - 2 b_1 b_2 \cos \varphi} \right)
               \sin^2 \varphi
             \, \mathrm{d} \varphi
           \, \mathrm{d} b_2
         \, \mathrm{d} b_1.
       \end{multlined}
  \end{aligned}
\end{equation}

\section{Proton}

Proton dipole form factor:
\begin{equation}
  F_p(Q^2)
  = \frac{1 + \frac{(\mu_p - 1) \tau}{1 + \tau}}
         {\left( 1 + \frac{Q^2}{\Lambda^2} \right)^2},
\end{equation}
where $\tau = Q^2 / 4 m_p^2$, $m_p$ is the proton mass, $\mu_p$ is the proton
magnetic moment (ratio to nuclear magneton).

Equivalent photon spectrum of a proton with the form factor approximated with
the dipole formula:
\begin{equation}
  \begin{split}
    n_p(\omega)
   &= \frac{\alpha}{\pi \omega}
      \left\{
          \left( 1 + 4 a - 2 (\mu - 1) \frac{a}{b} \right)
          \ln\left( 1 + \frac{1}{a} \right)
   \right. \\ &\qquad \left. {}
        + \frac{\mu - 1}{(b - 1)^4}
          \left(
            \frac{\mu - 1}{b - 1} (1 + 4 a + 3 b) - 2 \left( 1 + \frac{a}{b} \right)
          \right)
          \ln \frac{a + b}{a + 1}
        - \left( 4 + \frac{1}{a + 1} + \frac{1}{6 (a + 1)^2} \right)
   \right. \\ &\qquad \left. {}
        + 2 (\mu - 1) \left[
              \left( \frac{a}{b} + \frac{1 + a/b}{(b - 1)^3} \right)
              \frac{1}{a + 1}
            + \frac12
              \left( \frac{a}{b} - \frac{1 + a/b}{(b - 1)^2} \right)
              \frac{1}{(a + 1)^2}
            + \frac13
              \left( \frac{a}{b} + \frac{1 + a/b}{b - 1} \right)
              \frac{1}{(a + 1)^3}
         \right]
   \right. \\ &\qquad \left. {}
       - (\mu - 1)^2
         \left[
           \frac{1}{(b - 1)^4}
           + \frac{1 + 3 a + 2 b}{(a + 1) (b - 1)^4}
           - \frac{1 + 2 a + b}{2 (a + 1)^2 (b - 1)^3}
           + \frac{1}{3 (a + 1)^2 (b - 1)^2}
         \right]
     \right\}
    \\
    &= \frac{\alpha}{\pi \omega}
       \left\{
           \left( 1 + 4 u - 2 (\mu_p - 1) \frac{u}{v} \right)
           \ln \left( 1 + \frac{1}{u} \right)
    \right. \\ &\qquad \left. {}
         + \frac{\mu_p - 1}{(v - 1)^4} \left[
               \frac{\mu_p - 1}{v - 1} (1 + 4 u + 3 v)
             - 2 \left( 1 + \frac{u}{v} \right)
           \right]
           \ln \frac{u + v}{u + 1}
         - \frac{24 u^2 + 42 u + 17}{6 (u + 1)^2}
    \right. \\ &\qquad \left. {}
         + (\mu_p - 1) \,
           \frac{
             6 u^2 (v^2 - 3 v + 3) + 3 u (3 v^2 - 9 v + 10) + 2 v^2 - 7 v + 11
           }{
             3 (u + 1)^2 (v - 1)^3
           }
    \right. \\ &\qquad \left. {}
         - (\mu_p - 1)^2 \,
           \frac{
             24 u^2 + 6 u (v + 7) - v^2 + 8 v + 17
           }{
             6 (u + 1)^2 (v - 1)^4
           }
       \right\},
    \\
  \end{split}
\end{equation}
where
\begin{equation}
  u = \left( \frac{\omega}{\Lambda \gamma} \right)^2, \ 
  v = \left( \frac{2 m_p}{\Lambda} \right)^2.
\end{equation}

Equivalent photon spectrum at distance $b$ of a proton with the form factor
approximated with the dipole formula:
\begin{multline}
  \begin{split}
    n(b, \omega)
    &= \frac{\alpha}{\pi^2 \omega}
       \left[
           \frac{\omega}{\gamma} K_1 \left( \frac{b \omega}{\gamma} \right)
         - \left(
               1
             + \frac{(\mu_p - 1) \frac{\Lambda^4}{16 m_p^4}}{
                 \left( 1 - \frac{\Lambda^2}{4 m_p^2} \right)^2
               }
           \right)
           \sqrt{\Lambda^2 + \frac{\omega^2}{\gamma^2}}
           \, K_1 \left( b \sqrt{\Lambda^2 + \frac{\omega^2}{\gamma^2}} \right)
    \right. \\ &\qquad \left.
         + \frac{(\mu_p - 1) \frac{\Lambda^4}{16 m_p^4}}{
             \left( 1 - \frac{\Lambda^2}{4 m_p^2}  \right)^2
           }
           \sqrt{4 m_p^2 + \frac{\omega^2}{\gamma^2}}
           \, K_1 \left( b \sqrt{4 m_p^2 + \frac{\omega^2}{\gamma^2}} \right)
    \right. \\ &\qquad \left.
         - \frac{1 - \frac{\mu_p \Lambda^2}{4 m_p^2}}
                {1 - \frac{\Lambda^2}{4 m_p^2}}
         \cdot
           \frac{b \Lambda^2}{2}
           \, K_0 \left( b \sqrt{\Lambda^2 + \frac{\omega^2}{\gamma^2}} \right)
       \right]^2.
  \end{split}
\end{multline}

Empirical formula for the probability to avoid non-electromagnetic interactions
in an ultraperipheral collision of two protons~\cite{hep-ph-0608271}:
\begin{equation}
  P(b) = \left( 1 - \mathrm{e}^{-\frac{b^2}{2B}} \right)^2,
\end{equation}
where~\cite{1112.3243}
\begin{equation}
  B = B_0 + 2 B_1 \ln(E / E_0) + 4 B_2 \ln^2 (E / E_0),
\end{equation}
$B_0 = 12~\text{GeV}^{-2}$, $B_1 = -0.22 \pm 0.17~\text{GeV}^2$, $B_2 = 0.037
\pm 0.006~\text{GeV}^{-2}$, $E$ is the collision energy, $E_0 = 1$~GeV.

Photon-photon luminosity in an ultraperipheral collison of two protons:
\begin{equation}
  \begin{gathered}
    \begin{multlined}
      \frac{\mathrm{d} L_\parallel}{\mathrm{d} s}
      = \frac{\pi^2}{2}
        \int\limits_0^\infty b_1 \, \mathrm{d} b_1
        \int\limits_0^\infty b_2 \, \mathrm{d} b_2
        \int\limits_{-\infty}^\infty \mathrm{d} y
        \, n \left( b_1, \tfrac{\sqrt{s}}{2} \, \mathrm{e}^y \right)
        \, n \left( b_2, \tfrac{\sqrt{s}}{2} \, \mathrm{e}^{-y} \right)
        \\  \times
        \left\{
            1
          - 2 \mathrm{e}^{-\frac{b_1^2 + b_2^2}{2 B}}
            \left[
                I_0 \left( \frac{b_1 b_2}{B} \right)
              + I_2 \left( \frac{b_1 b_2}{B} \right)
            \right]
          + \mathrm{e}^{-\frac{b_1^2 + b_2^2}{B}}
            \left[
                I_0 \left( \frac{2 b_1 b_2}{B} \right)
              + I_2 \left( \frac{2 b_1 b_2}{B} \right)
            \right]
        \right\},
    \end{multlined}
    \\
    \begin{multlined}
      \frac{\mathrm{d} L_\perp}{\mathrm{d} s}
      = \frac{\pi^2}{2}
        \int\limits_0^\infty b_1 \, \mathrm{d} b_1
        \int\limits_0^\infty b_2 \, \mathrm{d} b_2
        \int\limits_{-\infty}^\infty \mathrm{d} y
        \, n \left( b_1, \tfrac{\sqrt{s}}{2} \, \mathrm{e}^y \right)
        \, n \left( b_2, \tfrac{\sqrt{s}}{2} \, \mathrm{e}^{-y} \right)
        \\  \times
        \left\{
            1
          - 2 \mathrm{e}^{-\frac{b_1^2 + b_2^2}{2 B}}
            \left[
                I_0 \left( \frac{b_1 b_2}{B} \right)
              - I_2 \left( \frac{b_1 b_2}{B} \right)
            \right]
          + \mathrm{e}^{-\frac{b_1^2 + b_2^2}{B}}
            \left[
                I_0 \left( \frac{2 b_1 b_2}{B} \right)
              - I_2 \left( \frac{2 b_1 b_2}{B} \right)
            \right]
        \right\},
    \end{multlined}
  \end{gathered}
\end{equation}
where $I_0(x)$, $I_2(x)$ are the modified Bessel function of the first kind.

\newcommand{\arxiv}[1]{\href{http://arxiv.org/abs/#1}{arXiv:\nolinebreak[3]#1}}
\begin{thebibliography}{99}
  \bibitem{hep-ph-0608271}
  L.~Frankfurt, C.~E.~Hyde-Wright, M.~Strikman, C.~Weiss.
  Generalized parton distributions and rapidity gap survival in exclusive diffractive $p p$ scattering.
  Phys.Rev.~D75, 054009 (2007).
  \href{http://arxiv.org/abs/hep-ph/0608271}{arXiv:\nolinebreak[3]hep-ph/0608271}

  \bibitem{1112.3243}
  V.~A.~Schegelsky, M.~G.~Ryskin.
  Diffraction cone shrinkage speed up with the collision energy.
  Phys.Rev.~D85, 094024 (2012).
  \arxiv{1112.3243}
\end{thebibliography}



\end{document}
